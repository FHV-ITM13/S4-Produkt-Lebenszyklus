\chapter{Einführung}

Productlifecyclemanagment, oder kurz PLM, ist der Versuch alle Daten, die im Lebenszyklus eines Produktes anfallen, zu verwalten. Dabei überlappen manche Funktionalitäten mit denen eines Versionskontrollsystems (VCS), welches in der reinen Softwareentwicklung verwendet wird.

PLM kann in die drei folgenden Bereiche eingeteilt werden:

\begin{itemize}
    \item Product Development
    \item Product Manufacturing
    \item Product Ownership
\end{itemize}

Product Development und Product Manufacturing beschäftigen sich mit der Produktnummer bzw. um das einzelne Produkt, wohingegen Product Ownership eine ganze Produktserie behandelt.

Produkte werden auch noch in Produktfamilien und Produktvarianten eingeteilt. Eine Produktfamilie fasst ähnliche Produktvarianten zusammen, welche sich nur in einzelnen Attributen voneinander unterscheiden.

Produkte sind anhand ihrer Produktdefinitionsdaten nachbaubar, würde in Software also dem Quellcode entsprechen. Im Gegensatz dazu existieren noch die Produktspezifikationsdaten, welche nur das Verhalten des Produktes beschreiben, und somit den Schnittstellendefinitionen einer Software entspricht. Bei der Entwicklung dieser Daten sollten auch die bereits existierenden Normen anderer Firmen oder Organisationen beachtet werden.

\section{Lebenszyklus}

Der Lebenszyklus eines Produktes hängt ganz wesentlich von der Industrie ab. Die Industrien werden in verschiedene Typen eingeteilt.

\paragraph{Process Production}
Hier läuft ein Prozess durchgehen ab, wie zum Beispiel beim Verarbeiten von Rohöl in einer Pipeline.

\paragraph{Batch Production}
Bei diesem Typ werden sehr große Mengen, wie zum Beispiel ganze Paletten, von einem Produkt produziert.

\paragraph{Embedded Systems}
Hier ist die Verwendung einer reinen Versionskontroll etwas schwierig, da die Software Teil eines physikalischen Produktes ist. Wenn man diese Produktdaten zusammen verwalten will, muss man ein PLM-System einsetzen.

\paragraph{Construction}
Das Bauwesen stellt einen Sonderfall dar, da das Produkt, typischerweise ein Gebäude, vor Ort erstellt werden muss.
