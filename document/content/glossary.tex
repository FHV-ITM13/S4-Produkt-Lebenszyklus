\chapter{Glossar}

\paragraph{PLM - Product Lifecycle Management}
ist der Versuch alle Daten, die im Lebenszy- klus eines Produktes anfallen, zu verwalten.

\paragraph{VCS - Version Control System}
Eine Versionsverwaltung ist ein System, das zur Erfassung von Änderungen an Dokumenten oder Dateien verwendet wird. \footnote{vgl.: \url{http://de.wikipedia.org/wiki/Versionsverwaltung} (12.06.2014)}

\paragraph{Process Production}
Hier läuft ein Prozess durchgehen ab, wie zum Beispiel beim Verarbeiten von Rohöl in einer Pipeline.

\paragraph{Batch Production}
Bei diesem Typ werden sehr große Mengen, wie zum Beispiel ganze Paletten, von einem Produkt produziert.

\paragraph{Embedded Systems}
Hier ist die Verwendung einer reinen Versionskontroll etwas schwierig, da die Software Teil eines physikalischen Produktes ist. Wenn man diese Produktdaten zusammen verwalten will, muss man ein PLM-System einsetzen.

\paragraph{Construction}
Das Bauwesen stellt einen Sonderfall dar, da das Produkt, typischerweise ein Gebäude, vor Ort erstellt werden muss.

\paragraph{Objektorientierung}
Unter Objektorientierung, kurz OO, versteht man eine Sichtweise auf komplexe Systeme, bei der ein System durch das Zusammenspiel kooperierender Objekte beschrieben wird. \footnote{vgl.: \url{http://de.wikipedia.org/wiki/Objektorientierung} (12.06.2014)}

\paragraph{UML - Unified Modeling Language}
Die Unified Modeling Language (Vereinheitlichte Modellierungssprache), kurz UML, ist eine grafische Modellierungssprache zur Spezifikation, Konstruktion und Dokumentation von Software-Teilen und anderen Systemen. \footnote{vgl.: \url{http://de.wikipedia.org/wiki/UML} (12.06.2014)}

\paragraph{Klassen}
Eine Klasse ist in UML eine Repräsentation von einer Gruppe von Dingen mit denselben Charakteristiken und Verhalten.

\paragraph{Spezialisierung und Generalisierung}
Die Beziehung zwischen Klassen kann so gesehen werden, dass jedes Objekt einer spezielleren Klasse auch ein Objekt einer genereleren Klasse ist.

\paragraph{Assoziation}
Eine Assoziation ist eine Verbindung zwischen 2 oder mehreren Objekten und wird als Linie zwischen den dazugehörigen Klassen dargestellt.

\paragraph{SysML Anforderungsdiagram}
Diese Diagramme gibt es in graphischer und tabellarischer Form sowie als Metamodel.

\paragraph{UML Objekt Diagramm}
Das Objektdiagramm ist ein Strukturdiagramm und umfasst ausprägungen von Klassen und Associationen.

\paragraph{Petri Netze}
Mathematische Modelierung von Verteilten Systemen.

\paragraph{Produktmanagment}
Produktmanagement umfasst die Planung, Entwicklung, Produktion, Vermarktung und das Outphasing von Produkten. Unternehmen mit einem Produktmanagement teilen ihr gesamtes Produktprogramm/-sortiment in Produkte und/oder Produktgruppen auf. Für die Aufgaben und den Erfolg des Produktmanagements ist der Produktmanager verantwortlich. \footnote{vgl.: \url{http://de.wikipedia.org/wiki/Produktmanagement} (12.06.2014)}

\paragraph{Anforderungsmanagement}
Anforderungsmanagement ist ein Teilgebiet des Requirements Engineerings sowie ein Teilgebiet der Business-Analyse und eine Managementaufgabe für die effiziente und fehlerarme Entwicklung komplexer Systeme. \footnote{vgl.: \url{http://de.wikipedia.org/wiki/Anforderungsmanagement} (12.06.2014)}

\paragraph{Item Spezifikation}
Resultat einer technischen Entwicklung, unabhängig von dessen Gebrauch als Teil eines größeren Produktes und Zubehör für die Produktion.

\paragraph{Item Kontext}
Wichtige Informationen um die Spezifikation richtig zu interpretieren.

\paragraph{Release}
Die fertige und veröffentlichte Version einer Software (bzw. Produkt) wird als Release bezeichnet. Damit geht eine Veränderung der Versionsbezeichnung, meist ein Hochzählen der Versionsnummer einher. \footnote{vgl.: \url{http://de.wikipedia.org/wiki/Entwicklungsstadium_(Software)} (12.06.2014)}

\paragraph{Version}
Veröffentlichtes Design, egal ob limitierte Veröffentlichung (limitierte erhalter) oder ganz öffentlich.

\paragraph{Versionsnummern}
Versionsnummern sind Identifier innerhalb des Item Kontext. Sie müssen Global einheitlich sein für Item-Identifier und Versionsnummer.

\paragraph{Varianten-Management}
Das Variantenmanagement (oder Variantenverwaltung) ist ein ganzheitlicher Ansatz zur Optimierung der Variantenvielfalt eines Produkts oder einer Produktfamilie im Hinblick auf Gesamtkosten, Marktstrategie und Kundenzufriedenheit. \footnote{vgl.: \url{http://de.wikipedia.org/wiki/Variantenmanagement} (12.06.2014)}

\paragraph{Stückliste}
Eine Stückliste (englisch: parts list oder bill of materials (BOM)) ist eine strukturierte Anordnung von Objekten oder eine Liste von Bauteilen eines umfassenderen Objektes, insbesondere von Erzeugnissen (Produkten) oder Baugruppen bzw. eines Zusammenbaus. Eine Materialstückliste ist eine strukturierte Anordnung von Teilen oder Baugruppen, die zur Herstellung einer anderen übergeordneten Baugruppe benötigt werden. \footnote{vgl.: \url{http://de.wikipedia.org/wiki/Stückliste} (12.06.2014)}

\paragraph{Manufacturing Engineering}
Auf Basis von Verfügbarkeit (Produktionseinrichtungen, Ausrüstungen, Geräte, ...) wird die Instandhaltung und Anpassung des Produktionsprozesse ausgeführt.

\paragraph{Sales Order Validation}
Bestellungen werden vom Verkauf auf Grund von Produktspezifikationen/- definitionen und Informationen von PDM/PLM-Systemen gefiltert.

\paragraph{Production Planning}
Vorbereiten von detailierten und synchronisierten Abläufen für Produktionsstellen und Transporteinrichtungen auf Basis von Bestellungen sowie Informationen aus PDM/PLM-Systemen.

\paragraph{Materials Procurement}
Finale Entschdeiungen auf Basis von Materialanfragen von der Produktionsplanung.

\paragraph{Manufacturing}
Die Qualitätssicherung der erhalten und produzierten Materialien sowie deren Lieferung wird auf Grund der Produktspezifikationen/-definitionen getroffen.

\paragraph{Support}
Der Support ist eine lösungsorientierte Beratungstätigkeit, beispielsweise in einem Callcenter oder Helpdesk. Insbesondere in der IT-Branche ist \"Support\" ein übliches Synonym für die Kundenbetreuung. \footnote{vgl.: \url{http://de.wikipedia.org/wiki/Support_(Dienstleistung)} (12.06.2014)}

\paragraph{External Supply Chain}
Platzhalter für alle Materiallieferaktivitäten die extern ablaufen.

\paragraph{Produktkonfiguration}
Informationen über den konkreten aktuellen Status bzw. die konkrete aktuelle Konfigurationen über das Produkt.

\paragraph{Produktinstandhaltung und -überholung}
Identifikation der Komponenten die angepasst und ersetzt werden müssen und Dokumentation der Änderungen.

\paragraph{Produkteinstellung und -außerbetriebnahme}
Identifikation der Komponenten die entfernt/ausgetauscht oder modifiziert/geschützt werden müssen. 

\paragraph{Produktdeinstallation und -reinstallation}
Nach der Außerbetriebnahme folgt die Deinstallation, Entfernen und zerlegen von physikalischen Komponenten um diese transportierbar zu machen. Nach dem Transport wird die zur Reinstallation übergegangen und gegebenenfalls werden Anpassungen (z.B. auf Grund von neuen Entwicklungen) vorgenommen. 

\paragraph{Produktzerstörung}
Hierbei werden physikalische Komponenten entfernt und zerlegt, allerdings kann hierbei die Integrität der Komponenten ignoriert/vernachlässigt werden. 

\paragraph{Normen / Normung}
Normung bezeichnet die Formulierung, Herausgabe und Anwendung von Regeln, Leitlinien oder Merkmalen durch eine anerkannte Organisation und deren Normengremien. Sie sollen auf den gesicherten Ergebnissen von Wissenschaft, Technik und Erfahrung basieren und auf die Förderung optimaler Vorteile für die Gesellschaft abzielen. \footnote{vgl.: \url{http://de.wikipedia.org/wiki/Normung} (12.06.2014)}

\paragraph{UseCase - Anwendungsfall}
Ein Anwendungsfall (engl. use case) bündelt alle möglichen Szenarien, die eintreten können, wenn ein Akteur versucht, mit Hilfe des betrachteten Systems ein bestimmtes fachliches Ziel (engl. business goal) zu erreichen. Er beschreibt, was inhaltlich beim Versuch der Zielerreichung passieren kann und abstrahiert von konkreten technischen Lösungen. Das Ergebnis des Anwendungsfalls kann ein Erfolg oder Fehlschlag/Abbruch sein. \footnote{vgl.: \url{http://de.wikipedia.org/wiki/Anwendungsfall} (12.06.2014)}

\paragraph{STEP}
Step ist ein Internationale Standard für CAD / PDM Daten austausch. Ausserdem bietet es standardisierte APIs für verschiedenste Zwecke.

\paragraph{CamelEAM}
CamelEAM ist ein OpenSource Facility Managment System, welches als Webapplikati- on in PHP und JavaScript geschrieben ist.


