\chapter{Einführung}

Productlifecyclemanagment, oder kurz PLM, ist der Versuch alle Daten, die im Lebenszyklus eines Produktes anfallen, zu verwalten. Dabei überlappen manche Funktionalitäten mit denen eines Versionskontrollsystems (VCS), welches in der reinen Softwareentwicklung verwendet wird.

PLM kann in die drei folgenden Bereiche eingeteilt werden:

\begin{itemize}
    \item Product Development
    \item Product Manufacturing
    \item Product Ownership
\end{itemize}

Product Development und Product Manufacturing beschäftigen sich mit der Produktnummer bzw. um das einzelne Produkt, wohingegen Product Ownership eine ganze Produktserie behandelt.

Produkte werden auch noch in Produktfamilien und Produktvarianten eingeteilt. Eine Produktfamilie fasst ähnliche Produktvarianten zusammen, welche sich nur in einzelnen Attributen voneinander unterscheiden.

Produkte sind anhand ihrer Produktdefinitionsdaten nachbaubar, würde in Software also dem Quellcode entsprechen. Im Gegensatz dazu existieren noch die Produktspezifikationsdaten, welche nur das Verhalten des Produktes beschreiben, und somit den Schnittstellendefinitionen einer Software entspricht. Bei der Entwicklung dieser Daten sollten auch die bereits existierenden Normen anderer Firmen oder Organisationen beachtet werden.

\section{Lebenszyklus}

Der Lebenszyklus eines Produktes hängt ganz wesentlich von der Industrie ab. Die Industrien werden in verschiedene Typen eingeteilt.

\paragraph{Process Production}
Hier läuft ein Prozess durchgehen ab, wie zum Beispiel beim Verarbeiten von Rohöl in einer Pipeline.

\paragraph{Batch Production}
Bei diesem Typ werden sehr große Mengen, wie zum Beispiel ganze Paletten, von einem Produkt produziert.

\paragraph{Embedded Systems}
Hier ist die Verwendung einer reinen Versionskontroll etwas schwierig, da die Software Teil eines physikalischen Produktes ist. Wenn man diese Produktdaten zusammen verwalten will, muss man ein PLM-System einsetzen.

\paragraph{Construction}
Das Bauwesen stellt einen Sonderfall dar, da das Produkt, typischerweise ein Gebäude, vor Ort erstellt werden muss.

\chapter{Spezifizierungstechniken}

\section{Objektorientierte Grundlagen}
Alles ( z.B. ein Fenster, eine Person) kann ein Objekt sein. Alle Objekte verfügen zudem über verschiedene Charakteristiken (z.B. Gewicht, Größe) und Verhalten (z.B. offen, geschlossen). Objekte die die selben Arten von Charakteristiken, Verhalten und Constraints haben, können in Klassen zusammengefasst werden.

\subsection{Klassen}
Klassen kann man sich im Informatikbereich als eine Art Bauplan für Objekte vorstellen. Im Prinzip sind Klassen selbst auch wieder Objekte und können reale Dinge oder computer-interne Representationen sein. Ein Objekt kann zu mehr als einer Klasse gehören (Mehrfachvererbung).

\paragraph{Spezialisierung und Generalisierung}
Die Beziehung zwischen Klassen kann so gesehen werden, dass jedes Objekt einer spezielleren Klasse auch  ein Objekt einer genereleren Klasse ist.

\begin{itemize}
	\item Multiple Klassifizierung
	\item Überlappende Klassifizierung
	\item Dynamische Klassifizierung
\end{itemize}

\section{UML Klassendiagramm}
UML ermöglicht verschiedene Sichtweisen auf ein und dasselbe Model. Je nach Programmiersprache gibt es angepasste Varianten von UML die nur das als UML zulassen was sich auch in der jeweiligen Programmiersprache wirklich umsetzen lässt.

\subsection{Klassen}
Eine Klasse ist in UML eine Repräsentation von einer Gruppe von Dingen mit denselben Charakteristiken und Verhalten. Operationen von Klassen sind die Definition eines Verhaltes einer Klasse.

\begin{itemize}
	\item Die UML-Spezifikation sagt nichts darüber aus, welche Art von Klassifizierung verwendet werden darf / kann.
	\item UML sagt auch nichts darüber aus welche Art der Konfliktlösung bei Mehrfachverwendung von selben Operationsnamen oder Attributnamen verwendet wird (z.B. bei Diamantenproblem). Darum ist die Forderung nach einer expliziten Neudefinierung eines Attributs / einer Operation möglich.
\end{itemize}

\subsection{Assoziation}
Eine Assoziation ist eine Verbindung zwischen 2 oder mehreren Objekten und wird als Linie zwischen den dazugehörigen Klassen dargestellt. Verbindungen können auch auf die selbe Klasse verweisen um mehrere Instanzen einer Klasse miteinander zu verbinden. Assoziationen sind an sich auch wieder Klassen und bieten somit die Möglichkeit bei den Verbindungen selbst weitere Attribute / Operationen zu definieren.

\paragraph{Nie mehr als 2 Klassen miteinander Verbinden} da es zu Probleme mit den Kardinalitäten bzw. deren logischer Auflösung führen kann.

\paragraph{Navigation}
In welche Richtung eine Assoziation führt wird mit Pfeilen ausgedrückt - ohne Pfeile bedeutet es, dass die Assoziation in beide Richtungen führt.

\paragraph{Aggregation}
Eine Verbindung die Darstellt welche Klasse ein Teil / im Besitz einer anderen Klasse ist(leere Raute).

\paragraph{Composition}
Eine Komposition ist auch eine Verbindungsvariante zwischen Objekten, wobei ein Objekt nur so lange exisitieren kann, solange es von einem anderen Objekt besessen wird. Außerdem kann das Objekt nicht von mehr als einem anderen Objekt besessen werden (ausgefüllte Raute).

\paragraph{Dependency}
Dient der Dokumentierung irgend einer Verbindung ohne weitere Aussage darüber, welcher Art diese Verbindung ist (strichlierte Linie).

\section{Weitere Diagramm-Arten}
Neben herkömmlichem UML gibts es auch weitere Diagrammarten: 

\paragraph{SysML Anforderungsdiagram}
Diese Diagramme gibt es in graphischer und tabellarischer Form sowie als Metamodel.

\paragraph{UML Objekt Diagramm}
Das Objektdiagramm ist ein Strukturdiagramm und umfasst ausprägungen von Klassen und Assiciationen.

\section{UML Communication Diagramm}
Das Kommunikationsdiagramm ist ein Verhaltensdiagramm. Es zeigt eine bestimmte Sicht auf die dynamischen Aspekte des modellierten Systems und stellt Interaktionen grafisch dar, wobei der Austausch von Nachrichten zwischen Ausprägungen mittels Lebenslinien dargestellt wird.

\subsection{Lifeline}
Eine Lifeline ist in UML eine Art instance eines Objektes, allerdings enthält es keinen Status der Instance. Also besser gesagt ist es ein Platzhalter für ein konkretes Objekt.

\subsection{Konkreter Nachrichten Fluss}
Informationen in diesem Diagramtyp:

\begin{itemize}
\item Abfolge der Nachrichten
\item Richtung der Nachrichten
\item Typen der Nachricht (synchron, asynchron)
\end{itemize}

Eine andere Darstellungsform für dieses Diagramm ist das Sequenz Diagramm. Es enthält die selben Informationen, stellt diese aber anders dar.

\subsection{Abstrakter Nachrichten Fluss}
Informationen in diesem Diagramtyp:

\begin{itemize}
\item Nicht technische Darstellung
\item Richtung der Nachrichten
\item Typen der kommunizierten Nachrichten
\end{itemize}

\paragraph{Beispiel} Crawler für eine Suchmaschiene

\subsubsection{Beispiel aus den Folien}

\img{1}{document/graphics/abstract_information_flow.png}{Abstrakter Informations Fluss}{aif}

\paragraph{Notizen:}

\begin{itemize}
\item Marketing schickt anforderungen an ein Produkt an Product Development (PD)
\item
{
PD entwickelt das Produkt => Ergebniss Product Definition
\begin{itemize}
\item Materials = Material um das Auto zu bauen (Stückliste)
\item Equipment = Zubehör (Ablauf, Werkzeuge, Prozesse ...) um das Auto zu bauen
\end{itemize}
}
\item Product Definition geht an Manfacturing
\item PD -> MaterialSpecification wird an Products Logistics geleitet wen das Teil eingekauft wird
\item Manufacturing bestellt über MaterialsRequest
\item Manufactoring verbaut den Motor (input item)
\item Fals gewisse Teile doch selber (oder andere Firma) Produziert werden ManufactoringOrder
\item FacillityLogistics bekommt ManufactoringOrder und liefert eine Resource item (unterstützt den Prozess)
\item PD liefert ProductDefinition an Marketing für neues Produkt
\item Marketing liefert ProductSpecification an Sales um Order zu erstellen
\item Sales liefert Order an DistributionLogistics (DL)
\item DL liefert (input) an Manufactoring
\item
{
Stereotypen:
\begin{itemize}
\item Input wird verbracht (nach der verarbeitung nicht mehr da)
\item Output 
\item Resource ... Hilfsmittel um die Aufgabe zu erledigen
\item Control ... Steuert den Ablauf im Block
\end{itemize}
}
\end{itemize}

\paragraph{Beispiel: Kompilierung}

\begin{itemize}
\item Input ist der Source code (weg für den Kompiler)
\item Resource ist der Kompiler, BS
\item Control ist z.B. Compiler direktiven
\end{itemize}

\subsection{UML Aktivitäts Diagramm}
Beschreibt das Verhalten (Fluss), nicht die statischen Strukturen.

\paragraph{Anwendungen}

\begin{itemize}
\item Geschäftsprozesse
\item Systemprozesse
\item ...
\end{itemize}

\subsubsection{Petri Netze}
Mathematische Modelierung von Verteilten Systemen. Entwickelt 1962 con Carl Adam Petri. Grundlagen für Petri Netze sind gerichteten (bipartiten) Graphen:
\begin{itemize}
\item
{
Nodes (Knoten):
\begin{itemize}
\item Places: tragen Tokens in sich
\item
{
Transition: Forked oder Joined Kanten
\begin{itemize}
\item Input Place: Verbunden über Kante am Eingang
\item Output Place: Verbunden über Kante am Ausgang
\end{itemize}
}
\item Die Transition feuert, wenn jeder Input Place mindestens soviel Tokens wie das Gewicht der Kante besitzt.
\item Die Output Places erhalten nach feuern der Transition soviele Tokens, wie das Gewicht der verbindenden Kante
\end{itemize}
}
\end{itemize}

\img{0.8}{document/graphics/ActivityDiagram.png}{Activity Diagram}{ad}

\newpage

\paragraph{Activity}
Entspricht einem Verhalten im System und besteht aus mehereren Actions.

\img{0.3}{document/graphics/Activity.png}{Activity}{activity}

\paragraph{Action}
Eine Aktion ist ein einzelner Schritt in einer Activity. Muss aber nicht Atomar sein (kann also aus weiteren Activities und Actions bestehen)

\img{0.3}{document/graphics/Action.png}{Action}{action}

\paragraph{Start und Endknoten}
Start enthält genug Token für die Eingaben, Feuert am start der Aktivität. Wen ein Token den Ende Knoten erreicht ist die Activity zu ende.

\img{0.3}{document/graphics/ActivityFinalNode.png}{ActivityFinalNode}{final}
\img{0.3}{document/graphics/InitialNode.png}{InitialNode}{init}

\paragraph{Kanten}
Die Kanten des Diagramm besitzt einen Namen und weitere Inforamtionen wie den Fluss (Richtung, Bedinungen, weiteres Verhalten). Zusätzlich spezifiert wird die Kante über Stereotypes.

Weitere Informationen zu Kanten:
\begin{itemize}
\item Gewicht \{Geschwungenen Klammern\}
\item Guard (Filter oder Evaluationen) [Eckige Klammern]
\item Zuerst Guard dann Gewicht überprüfen
\end{itemize}

\img{0.3}{document/graphics/ControlFlow.png}{Knoten}{flow}

\paragraph{Ein und Ausgänge (In/Output Pins)}
Entspricht Parameter in der Softwareentwicklung. Damit kann bestimmt werden welche Eingaben welche Rollen spielen. Damit können verschiedenen Eingabeparameter verschiedene Verhalten zugewiesen werden.

\img{0.3}{document/graphics/InputPin.png}{InputPin}{input}
\img{0.3}{document/graphics/OutputPin.png}{OutputPin}{output}

\paragraph{Objekt Knoten}
Repräsentiert Objekte die in Aktionen verwendet werden.

\img{0.3}{document/graphics/ObjectNode.png}{ObjectNode}{object}

\paragraph{Parameter Knoten}
Erweiterte Version von Ein und Ausgängen

\img{0.3}{document/graphics/ActivityParameterNode.png}{ParameterNode}{parameternode}

\paragraph{Entscheidungsknoten}
Nur ein Ausgang wird gefeuert. Bedinung für jeden Ausgang an der jeweiligen Kante.

\img{0.3}{document/graphics/DecisionNode.png}{DecisionNode}{decision}

\paragraph{Merge}
Es reicht wenn ein Eingang gefeuert wurde, damit der Ausgang gefeuert wird.

\img{0.3}{document/graphics/MergeNode.png}{MergeNode}{merge}

\paragraph{Fork / Join}
Erzeugt / Synchronisiert Parallele Flüsse. 

\img{0.3}{document/graphics/ForkNode.png}{Fork / Join Node}{fork_join}

\subsubsection{Advanced Activity Modelling}
Einführung von Activity Partitions, also mehrere Swimlanes. Swimlanes können später noch weiter unterteilt werden (FHV kann im Laufe des Prozesses z.B. noch in einzelne Räume unterteilt werden). 

Der external Stereotype beschreibt einen Bereich, der nicht von uns konstruiert wird, sondern von außen vorgegeben ist.

Wenn auf eine große Menge von Daten die gleiche Aktion angewendet werden soll, dann können sogenannte Expansion Regions verwendet werden. Dazu gibt es optionale Stereotypen für die Verarbeitung in einem iterativen, parallelem oder einem Streaming-Prozess. Die Verwendung dieser Stereotypen ist zum Teil allerdings an gewisse Hardware gebunden.

Darüberhinaus gibt es auch noch für die Darstellung von Loops die Standard Loops und Loop Nodes. Bei den Loop Nodes werden drei Bereiche angegeben: Setup, Test und einen ausführenden Body.

Streaming Actions können Output generieren, während dem man einen Input erhaltet.

Central Buffer stellen Puffer als Knoten dar, was z.B. bei einer Produktionskette ganz nützlich sein kann. Eine Erweiterung dazu stellt der Data Store Node dar, auf welchen auch noch Daten gespeichert werden. Diese Daten können auch später noch einmal abgerufen werden.

\subsubsection{Statecharts}
In vielen Fällen hängt das Verhalten eines Objekts von seinem Zustand ab, und die Zusammenhänge zwischen diesen Zuständen werden in einem Statechart mit Zustandsübergangen gezeigt. Wenn ein Übergang zwischen zwei States nicht definiert ist, könnte man das als nicht erlaubter Übergang interpretieren.

\paragraph{Status}
Ein Status haltet immer eine invariante Bedingung, die sich nicht ändert, solange man den Status nicht wechselt. Diese Bedingung ist im Normalfall implizit definiert.

Es gibt verschiedene Verhalten bei einem Status. Am Anfang gibt es ein Eintrittsverhalten, welches ausgeführt wird, sobald man den Zustand betritt. Während dem Verharren in einem Zustand gibt es ein Verhalten, welches gültig ist, solange der Status aktiv ist. Zuletzt gibt es noch ein Austrittsverhalten, welches beim Verlassen des Status ausgeführt wird.

Status können auch verschachtelt, und in dieser Verschachtelung auch noch in Swimlanes angeordnet werden. Man kann auch Submaschinenstatus verwenden, um die mehrfache Verwendung des gleichen Status in verschiedenen Diagrammen zu verwenden.

\paragraph{Zustandsübergang}
An einem Zustandsübergang hängt ein Trigger, welcher beschreibt bei welchem Event der Übergang eintritt. Dazu gibt es auch noch einen Guard, also ein boolscher Ausdruck, welcher wahr sein muss um den Übergang zu erlauben. Zuletzt gibt es einen Effekt, der beim Zustandsübergang ausgeführt wird.

\paragraph{Pseudostatus}
Es gibt auch sogenannte Pseudostatus, allen voran der Startzustand. Allerdings ist der Endzustand kein Pseudozustand, da an diesem wirklich etwas ankommt.

Darüber hinaus gibt es auch noch einige andere Pseudostatus, wie z.B. der Entscheidungsknoten, das Forken und Joinen, das Zusammenführen usw.

\paragraph{Signal}
Dieser Teil ist vor allem für Embedded Software wichtig, damit kann man das Empfangen und Versenden von Signalen modellieren.

\chapter{Produktmanagment}
Jedes Produkt hat verschiedene Lebenszyklen:

\begin{itemize}
\item product business lifecycle
\item product engineering lifecycle
\item manufacturing engineereing lifecycle
\item product support lifecycle
\item operator lifecycle
\end{itemize}

Je nachdem in welchem Zyklus sich ein Produkt befindet hat es mehr oder weniger Auswirkungen auf Entscheidungen von Produktmanagern --> z.B. wegen Rentabilität | Break Even Point.

Wichtig im Bereich des Produktlebenszyklusmanagement sind die Bereiche Herstellungs- und  Produktionsprozesse. Hier kann unterschieden werden zwischen:

\begin{itemize}
\item Kontinuierlicher Produktion
\item Stapelproduktion
\item Diskrete Herstellung
	\begin{enumerate}
	\item Einzigartiges Produkt
	\item Produkt in geringem Umfang mit und ohne Varianten
	\end{enumerate}
\end{itemize}


\img{1}{document/graphics/classification_of_industries.PNG}{Klassifizierung von Industrien (Rot ist das Wichtigste)}{classification_of_industries}

Der Produktmanager ist das Bindeglied zwischen Produktion, Marketing, Vertrieb usw. und muss vermitteln. Außerdem sollte er deb vielen verschiedenen Breichen (Technik, Wirtschaftlich, Marketing) ein wenig Ahnung und auch Verständniss dafür haben.

\img{0.8}{document/graphics/product_management_responsibilities.PNG}{Produktmanagerverantwortungen}{produktmanagerverantwortungen}

Neben dem Wissen über die ganzen Teilgebieten (Produktion, Marketing, Verkauf) sollte ein guter Produktmanager auch wissen wieso Kunden ein Produkt kaufen oder auch nicht. (Beispiel v. Prof.)

\section{Anforderungsmanagementprozess}
Hier wird festgestellt ob Anforderungen gültig sowie konsisten untereinander (keine Wiedersprüche) sind. Außerdem wird überprüft ob es keine Duplikate gibt und die gesamte Sammlung der Anforderungen valide ist (nichts vergessen und nichts was entfernt werden sollte).

\img{1}{document/graphics/requirement_management_processes.PNG}{Anforderungsmanagement-Prozess}{requirement_management_processes}

\section{Item und Item-Context}

\paragraph{Item Spezifikation}
Resultat einer technischen Entwicklung, unabhängig von dessen Gebrauch als Teil eines größeren Produktes und Zubehör für die Produktion.

\paragraph{Item Kontext}
Wichtige Informationen um die Spezifikation richtig zu interpretieren.
\begin{itemize}
\item Detailierte Anforderungen
\item Verbindungen zu anderen Teilen
\item Klassifizierung
\item Zweck und Status des Items
\item Phase im Lebenszyklus
\end{itemize}

\img{1}{document/graphics/item_itemcontext.png}{Item and Item Context}{item_itemcontext}

\section{Item Entwicklungs Prozess}

Der Prozess wird für jede Stufe durchgeführt, daher wird der Prozess rekursive ausgeführt.

\imgangle{0.5}{90}{document/graphics/itemdev.png}{Item Development Process}{itemdev}

\section{Item Struktur}
Entspricht einer einfachen Stückliste für das Item. Es ist ausreichend für die Fertigungslogistik aber nicht für die Unterstützung mit einem Informationsmanagement System.

\begin{itemize}
\item Idee eines Artikels
\item UseCases sammeln
\item Liste von Funktionen 
\item Funktionen auf abstrakte Komponenten zuweisen
\item Suche nach bestehenden Komponenten
\item Fals keines gefunden werden kann starte einen neuen Design Prozess
\item Die UseCases auf dieser Ebene auf die abstrakte Komponente zugeordnet 
\end{itemize}

\newpage
\paragraph{Ideale Struktur}
Die M-N Beziehungen kommen daher, dass eine Funktion von meheren Komponenten und dass mehrere Komponenten eine Funktion erfüllen können.

\img{1}{document/graphics/itemstructure3.png}{Ideale Item Struktur}{itemstructure3}

\newpage
\paragraph{Reale Struktur}
Über die Beziehungsklasse (ItemStructureRelationship) kann ein Typ definiert werden. Zusätzlich kann über das Flag physicalContainment definiert werden, ob die Beziehung eine Komposition oder Aggregation ist.

\img{1}{document/graphics/itemstructure4.png}{Reale Item Struktur}{itemstructure4}

In einem gewissen Detailierungsgrad macht die Unstruktured Definitionen keinen Sinn mehr. Beispiel Heizkörper, jeder Heizkörper bekommt nun ein Instance Objekt. Dort findet sich wieder eine Stückliste mit dem Bestandteilen dieses einzelnen Items.

\newpage
\section{Item Änderungen}
Im Allgemeinen eine Sourcecode Verwaltung für Item Spezifikationen. Um zu einem späteren Zeitpunkt zu einem definierten "guten" Zustand zurückzukehren.

Beinhaltet Informationen wie:
\begin{itemize}
\item Anforderungen
\item Kontext
\item Spezifikationen
\item Definitionen
\item Änderungsverlauf
\end{itemize}

\paragraph{Struktur}
Der Item Kontext erhält einen Vorgänger und mehrere Nachfolger.

\img{1}{document/graphics/itemchange1.png}{Change Management Classes}{itemchange1}

\newpage
\img{1}{document/graphics/itemchange2.png}{Change Management Process}{itemchange2}

\paragraph{Ablauf}
\begin{itemize}
\item Problem Report analysieren: Gültigkeit, Duplikate, ...
\item Speichern in einer Datenbank
\item Kategorisieren (Verbesserungswünsche, Kritischer Fehler, ...)
\item Aufgrund von Marktpriorisierung (oder normale Priorisierung) ChangeRequests erstellen
\item Genemigung, Planung, Aufwandsschätzung
\item Arbeitsaufträge erstellen
\item Ausführung
\end{itemize}

\section{Item Naming}

Traditionelle Sicht laut ISO 10303-214:
\img{1}{document/graphics/itemnaming1.png}{Traditioneles Item Naming}{itemnaming1}

Zusätzlich kann jedes Teil mehrere Namen besitzen. Beispiel Personen:
\begin{itemize}
\item Namen
\item Studentennummer
\item Spitznamen
\item ...
\end{itemize}

Namen können verschiedene Typen und Kontexte haben. Kontext sind zum Beispiel Familien, Firmen, ... Gutes Beispiel für Kontext sind zum Beispiel Domains ilias.fhv.at -> ilias ist nur in fhv und fhv nur in at definiert.

\paragraph{Beispiel für Globale Identifikation: Identifikation von Bernd Wenzel}
\begin{enumerate}
\item Norm (UN - 0, ISO - 1, IEC - 2)
\item Land (AT - 43, DE - 49, ...)
\item Ort (Dornbirn - 6850, ...)
\item Gebäude (FHV S-Trakt - 792, ...)
\item Raum (Bernd Wenzel - 3007, ...)
\end{enumerate}

\img{1}{document/graphics/item_naming_optimal.png}{Endergebnis von Item Naming}{itemnaming2}

\newpage
\section{Item Release Versionierung}

\paragraph{Überlegungen}

\begin{itemize}
\item Verschiedene Versionen einer Sache beziehen sich auf verschiedene Designs 
\item
{Zwischen zwei verschiedenen Designs können folgende Beziehungen bestehen:
\begin{itemize}
\item Ein Design wurde von der anderen abgeleitet 
\item Beide Entwürfe wurden von der gleichen Person erstellt 
\item Beide Entwürfe wurden im Rahmen der gleichen Organisationen geschaffen 
\item Ein Design wird in Zukunft Design-Arbeit die andere ersetzen 
\item One-Design wird in der Produktion die andere ersetzen 
\end{itemize}
}
\end{itemize}

\img{1}{document/graphics/item_versioning_1.png}{Versionierung}{versioning1}

\newpage
\paragraph{Version}
Veröffentlichtes Design, egal ob limitierte Veröffentlichung (limitierte erhalter) oder ganz öffentlich.

\img{1}{document/graphics/item_versioning_2.png}{Versionierungs Statemachine für eien Item Kontext}{versioning2}

\paragraph{Versionsnummern}
Versionsnummern sind Identifier innerhalb des Item Kontext. Sie müssen Global einheitlich sein für Item-Identifier und Versionsnummer.

\img{1}{document/graphics/item_versioning_3.png}{Versionsnummern sind Identifier}{versioning3}

\newpage
\paragraph{Beispiel: Magic Draw 17.0.1}

\begin{itemize}
\item Item Identifier: Magic Draw existiert global im Unternehmen
\item Version 17 Identifier: 17 existiert in Magic Draw
\item Version 0 Identifier: 0 exisiert in Magic Draw 17
\item ...
\end{itemize}

\newpage
\section{Übung Teilenummer}

How can we represent the following information in an object model according to our naming und identification model.

\paragraph{The following identifiers refer to the same part}
\begin{itemize}
\item Bosch: 4711-17
\item BMW: 0814-12
\item Mercedes: 9876-14
\end{itemize}

\img{1}{document/graphics/identifier_excercise.png}{Lösung zur Aufgabe}{identifier_excercise}

\chapter{Produktentwicklung}

\section{Items als Produkte oder Teile (Parts)}
Die Begriffe "Produkt", "Part" und "Artikel" sind mehrdeutig verwendet 

\paragraph{Bedeutung 1}
Eine ItemSpezifikation eines Items werden die Erträge als eine Komponente des ganzen gesehen.

\paragraph{Bedeutung 2}
Verwendung als Instanz wie zum Beispiel ItemSpezifikation.

\paragraph{Hinweis}
Sehr oft beziehen sich "Produkt" und "Artikel" auf den gesamten, während "Teil" bezieht sich auf einen Artikel auf beliebiger Komponentenebene.

\img{0.7}{document/graphics/release_process.png}{Beispiel eines Release Prozesses (Automobil Industrie)}{release_process}

\paragraph{Überlegungen zu Abbildung \imgref{release_process}}
\begin{itemize}
\item{
Im Frozen Zustand:
\begin{itemize}
\item Die Ergebnisse werden Kollegen zur Verfügung gestellt.
\item So Stabil, dass der Prototyp kann erzeugt werden kann.
\item So Stabil, dass eine Testphase gestartet werden kann.
\end{itemize}
}
\item External Visibility: Kann das Design begutachtet werden? Ist das Design richtig um gesetzt?
\item Safety relevant: Ist das Teil Sicherheitsrelevant und Sicherheitstechnisch OK? 
\end{itemize}

\newpage
\section{Item als Zubehör (facilities), Ausrüstung (equipment) und Werkzeuge (tools)}
Die Begriffe "Facility", "Ausrüstung" und "Tool" sind mehrdeutig verwendet.

\paragraph{Bedeutung 1}
Als Teil des Ganzen.

\paragraph{Bedeutung 2}
Als eigenständige Instanz.

\paragraph{Hinweis}
\begin{itemize}
\item Sehr oft bezieht sich "Facility" auf der obersten Ebene in einer Produktionsressourcenstruktur.
\item Sehr oft bezieht sich "Equipment" auf unabhängig voneinander arbeit Ebenen in einer Produktionsressource Struktur 
\item Sehr oft bezieht sich "Tool" auf Kunden austauschbare Funktionseinheiten in einer Produktionsressourcenstruktur
\end{itemize}

\newpage
\section{Varianten}

\subsection{Automobil Industry}
In der Anfangszeit waren Autos Luxusgüter für den Adel. Gekauft wurde damals nur der Motor und der Antriebsstrang, die Karroserie wurde von sogenannten Wagenbauer verkauft.

\img{0.7}{document/graphics/automobil_industry.png}{Früherer Rolls Royce}{rolls_royce}

Henry Ford führte die Massenproduktion ein. Das war das Ende der individuellen Wagen.

\paragraph{Zitat:}
"You may have it any color you like, provided the color is black"

\img{0.7}{document/graphics/automobil_industry_1.png}{Ford Werk Verbindung Antriebsstrang und Karosserie}{ford_werk}

\paragraph{Sonderwünsche}
Die Kunden hatten aber weiterhin spezielle Wünsche. Dazu boten die Hersteller Spezialprodukte an. Dadurch hatten die Händler einige Probleme mit der Identität und der Qualität. Die Reaktion der Hersteller war dann, dass sie sogenannte PlusMinus Stücklisten anboten (ein bzw. ausbauen).

\newpage
\subsection{Varianten-Management}
Eingeführt ca. 1965. Dadurch bestellt der Kunde keine Artikelnummer mehr sondern ein Model mit einer bestimmten Anzahl von Optionen.

\begin{itemize}
\item Typische Anzahl von Optionen: 150
\item Nur geschlossene Kategorien
\item Bei Luxuriösen Autos / Trucks: 1200 mit offenen Kategorien
\end{itemize}

Für jede Variante kann nicht ein Identifier erzeugt werden. Weil mindestens 150 Optionen bei jeweils mindestens 2 Möglichketen $2^{150} = 10^{45}$.

Als Vergleich das Alter des Universums in Sekunden (ca): $2*10^{10}*4*10^{2}*10^5>10^{18}$

\paragraph{Verbreitung}
Flächendeckend in Deutschland und Schweden. Viele andere Europäische Hersteller limitieren das größtenteils auf Packete. Andere Länder ziehen kräftig nach und der Markt wächst stark.

\newpage
\subsection{Stückliste}
Traditionelle Stücklisten können Variantenmanagement kaum abbilden. Aber auch eine Erweiterung wie in Abbildung \imgref{bom1}

\img{0.7}{document/graphics/bom1.png}{Klassische Stückliste}{bom1}

\paragraph{Probleme dieser Struktur:}
\begin{itemize}
\item Dokumentation der Entwicklungsprozess 
\item Dokumentation von Anforderungen 
\item Alternative Lösungen 
\item Alternative Fertigungsprozesse 
\item{
Option gesteuert Alternativen 
\begin{itemize}
\item für Produkte 
\item für Prozesse
\end{itemize}
}
\end{itemize}

\newpage
Moderne Stücklisten sind Feingranular in Abbildung \imgref{bom2}. Sie ermöglichen das Anhängen von vielen Dokumenten für die Dokumentation, Alternative Lösungen, Prozesse und die Optionsgesteuerten Alternativen für Produkte bzw. Prozesse.

\img{1}{document/graphics/bom2.png}{Moderne Stückliste}{bom2}

\newpage
Für das Varianten Management verwenden viele Automobil Hersteller heutzutage folgende Stückliste aus Abbildung \imgref{bom3}. Diese Stückliste bietet die Möglichkeit Spezifikations Expressions, um Einschränkungen abbilden zu können. Diese Expressions sind aufgebaut mit einem Syntaxtree um auch komplexere Einschränkungen abbilden zu können.

\img{1}{document/graphics/bom3.png}{Automobil Industry Stückliste}{bom3}

Eine Flexiblere und mächtigere Struktur in Abbildung \ref{bom4}. In dieser Struktur lassen sich Optionsklassen durch andere Optionen freischalten. Zum Beispiel wird die Produktklasse Vergasser nur dann angeboten wen ein alter Motor gewählt wurde oder bestimmte Optionen können nur an bestimmten Produktionsstätten gebaut werden.

\img{1}{document/graphics/bom4.png}{More powerful and flexible structure}{bom4}

\newpage
\paragraph{Quelle von Optionen und Regeln}
\begin{itemize}
\item {
Marketing
\begin{itemize}
\item Kunden: Farben, Radio, Schiebedach, ...
\item Marktkontext: Kulturele Präferenzen in bezug auf Farbkombinationen, ...
\end{itemize}
}
\item {
Produktentwicklung
\begin{itemize}
\item Technische Optionen: Auf Teile-Ebene, ...
\item Technische Regeln: Kein Radio ohne Antenne, ...
\end{itemize}
}
\item {
Produktherstellung
\begin{itemize}
\item Herstellungs Optionen: Auf Teile-Ebene, ...
\item Herstellungs Regeln: Auswahl von Produktionsstandorten in Abhängigkeit von anderen Optionen, ...
\end{itemize}
}
\end{itemize}

Optionen stellen einen Kontext für jedes Element bereit. Dies ist in Abbildung \imgref{bom_options} zu entnehmen.

\img{1}{document/graphics/bom_options.png}{Stückliste + Produkt optionen}{bom_options}

\subsection{Produkt Optionen und Herstellungsprozesse bzw. Produkt kosten}
Die Struktur in Abbildung \imgref{bom_options} stellt beide Möglichkeiten schon zur verfügung.

\subsection{Zusammenfassung}
Das Variantenmanagement ermöglicht gigantische Möglichkeiten, aber es hat gewisse Limitierungen:

\begin{itemize}
\item Variable Abmessungen (wie in der Möbelindustry oder Sattelposition in der Automobilindustrie)
\item {
Verbesserung: Variation Management (knowledgement Engeneering)
\begin{itemize}
\item Unterstützung für alle Datentypen in unserer Ausdrücke
\item Generische Regelsprache ersetzt die begrenzte Option / Option Ausdruckssprache
\item \textbf{Achtung:} Komplexität einer vollständigen Programmiersprache
\end{itemize}
}
\end{itemize}

\chapter{Produktion}
Die Produktion ist abhängig von den Fertigungswerken, der Ausrüstung und den Werkzeugen bzw. Hilfsmittel. Bevor die Produktion starten kann, müssen diese Dinge existieren/vorhanden sein. Weiters hängt auch die Effiktivität des Fertigungsprozesses von dem Grad der Instandhaltung der Fertigungswerke, der Ausrüstung und der Werkzeuge bzw. Hilfsmittel ab.

\img{1}{document/graphics/manufacturing_process_overview.png}{Produktionsprozessüberblick}{manufacturing_process_overview}

\section{Effiktivität}
Effiktivität ist die Entscheidung einen Entwicklungsstand zu verwenden. Das "release level" hingegen ist die Entscheidung des Entwicklers, dass etwas fertig ist - wann es allerdings in den Release integriert wird, wird dabei nicht beeinflusst. 

\img{1}{document/graphics/effectivity.png}{Effektivität}{effectivity}

\section{Prozesse}
\paragraph{Manufacturing Engineering}
Auf Basis von Verfügbarkeit (Produktionseinrichtungen, Ausrüstungen, Geräte, ...) wird die Instandhaltung und Anpassung des Produktionsprozesse ausgeführt. Entscheidungen über die Effektivität von den Spezifikationen der Materialien werden auf Basis von Verfügbarkeit, Pläne des Marketings, Beobachtungen des Markts und der Probleme beim Produzieren getroffen. Aufgrund von Problemen beim Produktionsprozess werden Problemberichte erstellt.

\paragraph{Sales Order Validation}
Bestellungen werden vom Verkauf auf Grund von Produktspezifikationen/-definitionen und Informationen von PDM/PLM-Systemen gefiltert.

\paragraph{Production Planning}
Vorbereiten von detailierten und synchronisierten Abläufen für Produktionsstellen und Transporteinrichtungen auf Basis von Bestellungen sowie Informationen aus PDM/PLM-Systemen.

\paragraph{Materials Procurement}
Finale Entscheidung bezüglich:
\begin{itemize}
\item Fertigung oder Kauf 
\item Bestellung der Materialien
\item Verhandlungen mit externen Partnern über den Kauf von Materialien
\end{itemize}
auf Basis von Materialanfragen von der Produktionsplanung. Intern wird dann entschieden ob gekauft oder gefertigt wird.

\paragraph{Manufacturing}
Die Qualitätssicherung der erhalten und produzierten Materialien sowie deren Lieferung wird auf Grund der Produktspezifikationen/-definitionen getroffen. Zeitnahe Produktion von Materialien auf Basis des Produktionsplans sowie der Produktspezifikationen/-definitionen. 

\paragraph{External Supply Chain}
Platzhalter für alle Materiallieferaktivitäten die extern ablaufen.


\chapter{Support}

\section{Produktkonfiguration}
Produkt-Support-Ingeneure brauchen Informationen über den konkreten aktuellen Status bzw. die konkrete aktuelle Konfigurationen über das Produkt da dieser Status / diese Konfiguration von der Prduktdefintion des Produzenten abweichen kann. Kann gesetzliche Konsequenzen nach sich ziehen. 

\paragraph{Als Teil der Instandhaltung sollten Produkt-Support-Ingeneuere den aktuellen Status / die aktuelle Konfiguration dokumentieren (zumindestens auf dem Level der Teilenummer). Produktbesitzer sollten Seriennummern dokumentieren.}

\paragraph{Zu Support zählt unter anderem die initiale Installation sowie Deinstallation und Reinstallation.}

\img{1}{document/graphics/product_configuration.png}{Produktionkonfiguration}{product_configuration}

\paragraph{Sammeln initialer Konfigurationen}
Das Problem bzw. die Schwierigkeiten beim Sammeln initialer Konfigurationsdaten sind die entstehenden Kosten. Außerdem sollte dieses Sammeln sehr konsequent und von beginn an gemacht werden.

\section{Produktinstallation}

\begin{itemize}
\item Vorbereiten der physikalischen Umgebung
\item Installation des physikalischen Produkts (Abhängig von  Umgebung und Transport), entfernen der transportspezfischen Teile und dokumentieren
\item Installation der Supportdatenbank
\item Testen
\item Training der Benutzer
\end{itemize}

\section{Produktinstandhaltung und -überholung}
Identifikation der Komponenten die angepasst und ersetzt werden müssen und Dokumentation der Änderungen. Gefolgt von Testen und erneutem Training der Benutzer.

\section{Produkteinstellung und - außerbetriebnahme}
Identifikation der Komponenten die entfernt/ausgetauscht oder modifiziert/geschützt werden müssen. Auch hierbei müssen die Änderungen dokumentiert und die Tests erneut durchgeführt werden.

\section{Produktdeinstallation und -reinstallation}
Nach der Außerbetriebnahme folgt die Deinstallation, Entfernen und zerlegen von physikalischen Komponenten um diese transportierbar zu machen. Nach dem Transport wird die zur Reinstallation übergegangen und gegebenenfalls werden Anpassungen (z.B. auf Grund von neuen Entwicklungen) vorgenommen. Nach dem Abschluss des Wiederaufbaus sollte wieder getestet werden und natürlich sollte jeder Schritt auch dokumentiert werden.

\section{Produktzerstörung}
Hierbei werden physikalische Komponenten entfernt und zerlegt, allerdings kann hierbei die Integrität der Komponenten ignoriert/vernachlässigt werden. Auch dieser Schritt sollte dokumentiert werden um gegebenfalls belegen zu können dass alles korrekt abgelaufen ist.

\chapter{Produktbesitzer und Anwender}
Der Anwender hängt an der Distributionslogistik. Den Anwender interessiert nicht das Design. Weiters hat er kein Interesse an der Produktnummer, sondern nur an seiner Seriennummer.

Ein gutes Beispiel für die Trennung ist die Automobilindustrie, bei der z.B. mit Chiptuning durch reine Software die Leistung eines einzelnen Autos (Seriennummer) geändert werden kann.

\section{Kundensicht}
Die Firma, die als Kunde fungiert, stellt die Spezifikation für das jeweilige Produkt. Der Auftragnehmer kann dabei Hilfestellung geben, indem gewisse Informationen an den Auftraggeber weitergegeben werden. Aus diesem Austausch definiert der Auftragnehmer eine genauere Spezifikation, welche den Produktanbietern übergeben werden. Der Anbieter muss nicht zwingend auch der Hersteller sein, es kann sich dabei z.B. auch um einen Großhändler handeln.

Die detaillierten Information sind für den Auftraggeber nicht, oder nur von marginalem, Interesse, hauptsächlich müssen die detaillierten Spezifikationen erfüllt werden, und das Produkt funktionieren.

Mit dem Kunde sollte auch definiert werden welche Normen verwendet werden, da verschiedene Länder/Organisationen unterschiedliche Normen veröffentlichen. Für die verschiedenen Zwecke sollten eindeutige Normen bestimmt werden.

Problematisch sind dabei auch die Lücke zwischen den Datenmodelliereren und der Terminologie die von den Experten in der jeweiligen Domänen verwendet wird.

Folgende weitere Punkte sind für die Zusammenarbeit auch wichtig:
\begin{itemize}
\item Einigung auf einen gewissen Bereich der Arbeit (Use Cases)
\item Definition eines standardisierten Glossar von Klassen für das Projekt
\item Einigung auf standardisierte Attribute für die Begriffe im Glossar
\item Einigung auf die Maßeinheiten für die standardisierten Attribute
\item Überprüfung auf die Notwendigkeit von weiteren Attributen anhand von Use Cases
\item Abstimmung von zusätzlichen Attributen auf Grund von Anforderungen der Stakeholder
\item Einigung auf standardisierten/normierten Austauschformat für Informationen
\item Bei Möglichkeit \textbf{international anerkannte Normen} verwenden
\end{itemize}

\section{Anwenderprozessmodell}
Am Anfang muss man sich Gedanken über den Prozess machen, z.B. nach welchen Vorgehensweisen wird das Produkt hergestellt, welche Rohstoffe werden benötigt, was sind die Ergebnisse/Abfallprodukte des Prozesses.

Bei diesem Modell gibt es eine klare Trennung zwischen Prozess und Herstellung.

\imgangle{0.9}{90}{document/graphics/OwnerOperatorProcessModel.png}{Owner / Operator Process Model}{oopm}

Die Abbildung \imgref{oopm} gibt nur eine grobe Übersicht, hinter jedem Punkt hängt ein weiteres Diagramm, welches ungefähr die selbe Komplexität besitzt.

\section{Anwenderdatenmodell}
Hierbei werden die Änderungen über die Zeit am Produkt fest gehalten. Als Beispiel könnte man beim Auto das Wechseln von Einzelteilen oder sogar das Nachtanken nennen.

Dem Ganzen liegen zwei Prinzipien zu Grunde:

\begin{itemize}
\item Diese Änderungen können in einem flachgedrückten 3D-Raum über die Zeit dargestellt werden, welche umgangsprachlich auch Schlauchdiagramme genannt werden. Man kann sich das vorstellen, als ob die Objekte bei einer Bewegung durch den Raum kopiert werden.
\item Bei temporären Teilen wird ein Bauteil oder eine Funktion über einen gewissen Zeitraum betrachtet. Bei diesen Diagrammen werden die Prozesse definiert, und die Verwendung der einzelnen Bauteilen auf einem zweidimensionalen Graph augetragen werden. Weiters könnte man auch den verwendeten Platz eines Produkts über eine gewisse Zeit abgebildet werden. (Bilder Owner /Operator Data Model(2))
\end{itemize}

\img{1}{document/graphics/OwerOperatorDataModel.png}{Owner / Operator Data Model}{oodm}

Bei diesem Modell (Abbildung \imgref{oodm}) wird zuerst ein Thing definiert, was eine Abstraktion für alles ist. Es werden auch Klassen definiert, welche natürlich auch ein Thing darstellen.

Es gibt auch Relationships, welche selber Things darstellen, und zwei Things in eine Beziehung stellen. Relationships haben zwei Subklassen: Specialization und Classification. Eine Specialization setzt zwei Klassen in eine Beziehung, un deine Classification ein Thing zu einer Klasse.

PossibleIndivuals gibt verschiedene Möglichkeiten für Relationships dar. Eine Unterklasse davon stellt das PhysicalObject dar, welches wie folgt weiter unterteilt werden kann.

\begin{itemize}
\item MaterializePhysicalObject ist ein physisches Objekt, welches im Großen und Ganzen an seiner Materie definiert werden
\item FunctionalPhysicalObject wird durch seine Funktion definiert
\item SpacialLocation ist ein Punkt oder eine Menge von Punkten im Raum
\item Stream ist ein Fluss von Energie, Materie oder sonst irgendetwas
\end{itemize}

Die restlichen Möglichkeiten teilen sich wie folgt weiter:

\begin{itemize}
\item Ein anderes PossibleIndivual ist ein ActualIndiviual, wenn es in der Realität wirklich existiert.
\item Ein ArrangedIndivual kann seine Funktion während seiner Lebenszeit verändern
\item Ein WholeLifeIndividual hat über die gesamte Lebenszeit die gleiche Funktion
\item Eine Activity ist eine Tätigkeit, wie z.B. das Reden eines Dozenten, oder das Zusammenbauen eines Teiles
\end{itemize}

Things die Activities beeinflussen sind in folgender Liste beschrieben:

\begin{itemize}
\item CauseOfEvent stellt zwei Things in eine Relation, die ein Event verursacht
\item InvolvementByReference stellt die Verwendung Dingen in eine Beziehung
\item Recognition beschreibt die Verwendung ohne Veränderung
\end{itemize}

Mit diesem Modell kann die Veränderung in Objekten der Welt einschließlich der Welt selbst über die Zeit dokumentiert werden. Es ist das einzig bekannte Modell, welches solche Veränderungen über einen so breiten Rahmen dokumentieren kann.

\chapter{STEP}

\textbf{STEP ist ...}

\begin{itemize}
\item ein Internationale Standard für CAD / PDM Daten austausch
\item eine Sammlung von standardisierten Informationsmodellen und Modelierungssprachen
\item ein Datenaustauschformat
\item eine standardisierte API für Produktdaten
\item ein Produktmodel Framework
\end{itemize}

\section{Komponenten}

\begin{itemize}
\item {
Informationsmodelle
\begin{itemize}
\item Generische Resourcen und Application Interpreted Constructs (AIC)
\item Application Resources
\item Application Protokolls
\end{itemize}
}
\item Methodik und Werkzeuge
\item Implementation Forms
\item Conformance Testing Methodik
\end{itemize}

\section{Standardisierungs Zyklus}

\begin{itemize}
\item NWIP: New Work Item Proposal (Vorschlag)
\item NWI: New Work Item (Ausarbeitungs vorbereitung)
\item WD: Working Draft (Ausarbeitung in einer Gruppe)
\item CD: Committee Draft for Voting (Begutachtung => bei negativem Ergebnis Stufe zurück) 
\item DIS: Draft International Standard (Vorlage für Internationaler Standard)
\item FDIS: Final Draft International Standard (Finalisierte Fassung)
\item IS: International Standard
\end{itemize}

\section{Express - Modelierungssprache}

Standardisierte Modelierungssprache um Daten zu beschreiben. Es ist in der lage auch komplizierte Daten zu beschreiben (z.b. Conditions).

\newpage\section{Product Model Framework}

\begin{itemize}
\item Dünne Linien: Associationen
\item Dicke Linien: Generalisierung
\end{itemize}

\img{1}{document/graphics/product_modell_framework.png}{Product Model Framework}{pmf}

\section{Datenaustausch}

Für den Datenaustausch wird für Step folgendes Model verwendet.

\img{1}{document/graphics/data_exchange.png}{data exchange}{data_exchange}

Es wird mittels eines Präprozessors (muss selbst implmentiert werden) und ein STEP File genieriert. Dieses File kann mittels eines Postprozesors in das Ziel System geschrieben werden.

\img{1}{document/graphics/data_sharing.png}{data sharing}{data_sharing}

Daten gemeinsam verwenden kann über eine STEP Datenbank umgesetzt werden. Dafür werden für die jeweiligen System eine Konverter geschrieben der Daten aus der Datenbank in das Properitäre Format umgewandelt wird.

\newpage\section{Implementation Framework}

In Abbildung \imgref{implementation_framework} wird das Implementation Framework von Step aufgeführt.

\img{1}{document/graphics/implementation_framework.png}{Implementation Framework}{implementation_framework}

\newpage\section{Warum Step?}

\textbf{Step ist eine Technologie ...}

\begin{itemize}
\item für den austausch, Integration und Konsolidierung von Daten
\item für parallele Entwicklung
\item für Business Process Re-engineering 
\item für Produktlebenszyklus-Unterstützung 
\item für Unternehmen in ein agiler bewegen und wettbewerbsfähige Zukunft
\end{itemize}

\textbf{STEP} has got international support. \\
\textbf{STEP} has won industrial momentum.  \\
\textbf{STEP} is extensible.  \\
\textbf{STEP} is implementable.  \\
\textbf{STEP} works.
 
A bridge into current/future technology is being prepared 

\section{CamelEAM}
CamelEAM ist ein OpenSource Facility Managment System, welches als Webapplikation in PHP und JavaScript geschrieben ist. Wir haben versucht in der Vorlesung diese Webapplikation zu laufen zu bringen, nachdem OpenMAINT anscheinen noch in der Entwicklung ist.

Die Installationsanleitung ist nicht wirklich vollständig, dennoch haben wir es geschafft die Applikation soweit zu konfigurieren, das man sich einloggen kann. Das war nur nach Abschalten der Error-Meldungen möglich, da es sich hier anscheinend um nicht wirklich sauber programmierte Software handelt, die sehr viele Warnings verursacht. Wenn diese nun ausgegeben werden, werden fehlerhafte Requests abgesendet, und nicht mal der Login funktioniert.

Nach dem Abschalten dieser Fehlermeldungen funktioniert zwar der Login, allerdings triggern die ganzen Schaltflächen keine Aktionen, das Tool ist so also nicht zu verwenden. Unsere erste Vermutung ist, das JavaScript-Libraries fehlen, und dadurch kein korrekter Ablauf mehr möglich ist. Die notwendigen Files sind aber in dem Tar-Archiv, das in der Installationsanleitung erwähnt wird, nicht vorhanden.

